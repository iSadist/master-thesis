\chapter{Project Methodology}

When working in a project it is generally a good idea to work according to a predetermined project methodology model. This tends to make the work more effective and produce better quality. That is, more is produced during the same amount of time, the final product is more thought out, better teamwork etc.

Simply, if you have a plan at the start it is easier to stick to that plan and stay on the same path as intended, rather than shifting to another one. That is not to say that the plan cannot change, but if it does it does so controllably.

A historically good model has been the waterfall model. However, in the software industry this has changed lately with the agile methodology being more popular and it has proven to be very effective.

\section{Agile}
We have used the scrum method, which is an agile method.
Basically, the scrum model says that rather than planning a workload for 6 months forward or so it is better to work in short iterations. These iterations should be between 1 - 4 weeks depending on the team.
Instead of trying to estimate the time it will take to complete an entire project, the team is given a finite time frame and tries to complete as much as possible during that time. During this time, the team works very closely to the customer and project owner to make sure that they will get what they want and ask for.
For selecting items to work with for every sprint (explained further in section \ref{subSec:howweused}) the team keeps a backlog of items it wishes to complete during the project. Every sprint, these items or stories are picked out and included into the sprint and estimated in size. The stories themselves should be collected from customers, project owner, users and people connected to the product. This way, the team knows what the purpose of developing a certain thing is. If there is any doubt about a specific feature it should be easier to ask than to assume.

This is of course a simplified version of how the agile method works.
'Extreme Programming Pocket Guide' is a good book for anyone who wants to read more about agile methodology.\cite{extremeProgramming}

\section{How we used the agile methodology}
\label{subSec:howweused}
\textbf{Sprints} \\
For our planned work we have decided to work in two week period sprints. At each start of a sprint we pick out stories to focus on for the coming two weeks and try to estimate how long each of them will take.
On weekdays we start with a quick discussion about yesterdays work and what we are planning to do today. This is for everyone to be up to speed about the other person's work.
At the end of the sprint everything is reviewed and analyzed so to do even better the next sprint by correcting possible faults.\\

\noindent
\textbf{Story board} \\
For organizing our sprints we used a story board which contained an area for our backlog items and four rows for our stories during a sprint. The stories were broken into tasks which were placed on either one of the columns (Started, In Progress, Waiting, Completed).
Each story was estimated with a size which was a number in the fibonacci sequence. Each story was also divided into 5 parts (Started, Halfway done, Completed, Reviewed, Verified). These two numbers were multiplied and summed up with all the other stories to get how many points the sprints had. As we worked these points were subtracted and ultimately hit zero when we were done with everything. We plotted the progress on a chart as well for graphic representation. This is called a burndown.

\begin{figure}[hbtp]
\begin{center}
\includegraphics[width = 0.5\textwidth]{./Images/story_board.jpg} 
\caption{Story board used during the project}
\end{center}
\end{figure}

\noindent
\textbf{Close relationship with project owner} \\
We worked very close to the project owner by sitting on desks right across him. Whenever we had a questing we could simply raise our head and ask it.

\newpage