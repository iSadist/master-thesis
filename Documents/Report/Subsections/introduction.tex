\section{Introduction}
To first get an understanding of why this project was done, it is important for the reader to learn, not only the background of the fields studied, but also about the prospects of the technologies involved.
 At \hyperref[subsec:Background]{3.1} we go through the current and the potential use of these technologies; also what prospects the consumers and the industry has. Later on, at  \hyperref[subsec:PrevStud]{3.2} we go through some of the previous research that has been done in the fields that are studied in the report. This section is then rounded of at \hyperref[subsec:Goal]{3.3} with us telling you what we hoped to achieve with this project and a brief run-down of how.

\subsection{Background}
\label{subsec:Background}
Why is AR and Object recognition interesting, for the consumers and Jayway

The idea of Augmented Reality is to render virtual object in the real world. This usually requires hardware in the form of a camera and display, a processing unit and software. Common AR devices today are the HoloLens \url{https://www.microsoft.com/en-us/hololens}, Google's Glass \url{https://developers.google.com/glass/design/principles} and a vast amount of mobile devices, such as Apple's iPhone X \url{https://www.apple.com/lae/ios/augmented-reality/}. 

The general interest for AR has undeniably grown in recent years, with games and applications such as Niantic's Pokémon Go \url{https://www.pokemongo.com/en-us/} and IKEA's IKEA Place\url{https://itunes.apple.com/us/app/ikea-place/id1279244498?mt=8} successfully reaching vast amounts of users around the globe \textbf{Svårt att hitta användardata. Något att lösa framöver?}.

 But Augmented Reality has not only reached the casual users; the technology has also peaked an interest in several industries. One such project is Fieldbit's Fieldbit Hero \url{https://www.fieldbit.net/products/fieldbit-hero/}, a platform enabling technicians to get instant AR annotations to their AR devices from an engineer, allowing the technician to get visual instructions while simultaneously  being able to work hands-free. 



\subsection{Previous studies}
\label{subsec:PrevStud}
List of previous studies done in fields similar to our study. Also comment them and implement the citation in bibliography.




\subsection{The goal of this project}
\label{subsec:Goal}
What are we going to do and what do we hope to achieve with this project.

Our aim was to research and become familiar with these fairly new, but broad, subjects. 
\newpage
