\section{Introduction}
To first get an understanding of why this project was done, it is important for the reader to learn, not only the background of the fields studied, but also about the prospects of the technologies involved.
 At \hyperref[subsecBackground]{3.1} we go through the current and the potential use of these technologies; also what prospects the consumers and the industry has. Later on, at  \hyperref[subsecPrevStud]{3.2} we go through some of the previous research that has been done in the fields that are studied in the report. This section is then rounded of at \hyperref[subsecGoal]{3.3} with us telling you what we hoped to achieve with this project and a brief run-down of how.

\subsection{Background}
\label{subsecBackground}
Why is AR and Object recognition interesting, for the consumers and Jayway

The idea of Augmented Reality is to render virtual object in the real world. This usually requires hardware in the form of a camera and display, a processing unit and software. Common AR devices today are the HoloLens \cite{microsoft}, Google's Glass \cite{googleGlasses} and a vast amount of mobile devices, such as Apple's iPhone X \cite{appleAR}. 

The general interest for AR has undeniably grown in recent years, with games and applications such as Niantic's Pokémon Go \cite{pokemonGO} and IKEA's IKEA Place \cite{IKEAPlace} successfully reaching vast amounts of users around the globe \textbf{Svårt att hitta användardata. Något att lösa framöver}.

 But Augmented Reality has not only reached the casual users; the technology has also peaked an interest in several industries. One such project is Fieldbit's Fieldbit Hero \cite{fieldbit}, a platform enabling technicians to get instant AR annotations to their AR devices from an engineer, allowing the technician to get visual instructions while simultaneously  being able to work hands-free. 



\subsection{Previous studies}
\label{subsecPrevStud}
D. Chatzopoulos et al.(2017) describes the basics of Mobile Augmented Reality, its advancements and it's flaws. They estimate that MAR is the most promising field of mobile applications and that it will have a massive impact on how we interact with the real world. However, they also go through some of the current hiccoughs with the technology, such as bandwith limitation and the computing power required. 
\cite{MARS}

S.Gould et al.(2009) writes about a hierarchical model for joint object detection and image segmentation, where they basically tried to segment all objects in an image and classifies every pixel. 
\cite{NIPS2009_3766}


Y LeCun and M.A. Ranzato goes in-depth on the history and progress of deep learning, from the first machine learning model, the Perceptron, to future challenges with deep learning. They write about the use cases for machine learning and how models generally work. They also classify different kinds of models into separate categories. Also written about is areas such as what makes a good feature and how convolutional neural networks are constructed.
\cite{deepLearningTutorial}


\subsection{The goal of this project}
\label{subsecGoal}
What are we going to do and what do we hope to achieve with this project.

Our aim was to research and become familiar with these fairly new, but broad, subjects. 
\newpage
